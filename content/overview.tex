\section{\texorpdfstring{Tổng quan tình hình nghiên cứu}{overview}}
% Sơ lược, phân tích, đánh giá các công trình nghiên cứu nổi tiếng có liên quan đến đề tài. Nêu những vấn đề bức thiết cần phải giải quyết, chỉ ra những thiếu sót mà những nghiên cứu trước đây chưa giải quyết được.
Để tạo sinh mặt người đang nói, các công trình nghiên cứu tập trung chủ yếu vào vùng miệng. Bài nghiên cứu vào năm 2018 của Lele Chen \cite{chen2018} đưa ra phương pháp tạo sinh video vùng miệng của người đang nói với đầu vào là ảnh tĩnh của khuôn miệng và một đoạn âm thanh có chứa tiếng nói. Vào năm 2019, Lele Chen \cite{chen2019} và Vougioukas \cite{vougioukas2019} tiếp tục đưa ra phương pháp tạo sinh cả khuôn mặt người dựa vào ảnh tĩnh của khuôn mặt và đoạn âm thanh chứa tiếng nói. Năm 2020, Vougioukas \cite{vougioukas2020} đã cải tiến phương pháp tạo sinh mặt và cập nhật thêm hành động chớp mắt, Lele Chen \cite{chen2020} cũng đưa ra phương pháp mới để tạo sinh mặt hiệu quả hơn, tự nhiên hơn với việc di chuyển của vùng đầu trên khung hình.

Nhìn chung, các nghiên cứu này đã đưa ra các kiến trúc mạng hiệu quả để tạo sinh khuôn mặt cũng như các phương pháp, lập luận và chứng minh tính hiệu quả của các kiến trúc mạng được đề xuất. Mặc dù các thông số của thử nghiệm đưa ra là khá tốt, các nghiên cứu của Vougioukas vẫn chưa thể tạo ra chuyển động của đầu, kết quả tạo sinh của Vougioukas đôi khi không giữ được đặc trưng của ảnh. Nghiên cứu của Lele Chen năm 2020 \cite{chen2020} đã tạo ra chuyển động cho phần đầu dựa trên tiếng nói, nhưng khuôn mặt được tạo sinh vẫn còn có thể bị nhận ra qua các phép thử Turing, và chuyển động của đầu đôi khi vẫn chưa được tự nhiên, mạng cũng có cấu trúc rất phức tạp và đòi hỏi nhiều tài nguyên tính toán để có thể huấn luyện.
