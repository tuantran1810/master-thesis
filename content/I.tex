\section{\texorpdfstring{Giới thiệu đề tài}{Introduce}}
Bài toán tạo sinh dữ liệu dựa trên những nguồn dữ liệu có tính chất khác nhau đã và đang trở thành xu thế trong những năm trở lại đây. Đây là bài toán có tính cấp bách, mang lại giá trị cao về mặt kiến thức cho ngành trí tuệ nhân tạo nói riêng và giá trị về mặt kinh tế, công nghệ chung cho toàn xã hội xã hội. Bên cạnh đó, việc tạo sinh dữ liệu về con người đã đạt được những tiến bộ vượt bậc, đặc biệt là tạo sinh dữ liệu hình ảnh khuôn mặt người. Trong để tài này, mục đích nghiên cứu là: cho biết một vài dữ liệu về gương mặt của một người bất kỳ (hình ảnh, video ngắn) và một đoạn tiếng nói bất kỳ, tạo sinh hình ảnh khuôn mặt người đó đang nói đoạn tiếng nói đã cho một cách chân thực.

Ý nghĩa khoa học: Đóng góp cho sự phát triển chung của xu hướng tạo sinh dữ liệu mới dựa trên các tính chất của dữ liệu ban đầu. Việc tìm ra phương pháp giải quyết tốt bài toán sẽ tạo nên tảng để giải quyết những bài toán xa hơn, phức tạp hơn như: tạo sinh nửa người trên, tạo sinh toàn bộ cơ thể người, hay tạo sinh cả một bối cảnh trong phim. Đề tài giúp hiện thực, cải tiến các phương pháp hiện có trong các bài nghiên cứu gần đây, so sánh và cải tiến để cho ra kết quả tạo sinh tốt hơn, đóng góp thêm phương pháp mới cho việc tạo sinh ảnh. Đồng thời, các phương pháp tạo sinh dữ liệu cũng giúp làm giàu dữ liệu để huấn luyện, kiểm thử cho các mô hình học máy, học sâu khác.

Ý nghĩa thực tiễn: Giải quyết thành công vấn đề này đem lại giá trị to lớn về mặt công nghệ, kinh tế và xã hội. Chúng ta có thể tái hiện lại gương mặt người đang nói ở nhiều thứ tiếng khác nhau, tạo sinh khuôn mặt người đại diện trong các hội nghị trực tuyến, tích hợp vào các trò chơi điện tử để làm chúng trở nên chân thực hơn, truyền video trong điều kiện băng thông giới hạn, giả lập trợ lý ảo có hình dáng con người,... Đối với ngành truyền thông, nó có thể tạo ra biên tập viên ảo. Đối với ngành điện ảnh, giải trí, sáng tạo nội dung nó cũng có giá trị ứng dụng khi giúp giảm bớt áp lực lên khâu hóa trang, kỹ xảo.

Kiến trúc mạng Generative Adversarial Network \cite{gans_base} ra đời vào năm 2014 đã đánh dấu một bước chuyển mình mới cho ngành trí tuệ nhân tạo. Kiến trúc này giúp cho việc tạo sinh dữ liệu được thực hiện một cách hiệu quả và chính xác hơn. Dựa trên nền tảng đó, các nghiên cứu về việc tạo sinh ảnh gương mặt người cũng được tiến hành và ngày càng có những bước tiến mới. 

Để tạo sinh mặt người đang nói, các công trình nghiên cứu tập trung chủ yếu vào vùng miệng. Bài nghiên cứu vào năm 2018 của Lele Chen \cite{chen2018} đưa ra phương pháp tạo sinh video vùng miệng của người đang nói với đầu vào là ảnh tĩnh của khuôn miệng và một đoạn âm thanh có chứa tiếng nói. Vào năm 2019, Lele Chen \cite{chen2019} và Vougioukas \cite{vougioukas2019} tiếp tục đưa ra phương pháp tạo sinh cả khuôn mặt người dựa vào ảnh tĩnh của khuôn mặt và đoạn âm thanh chứa tiếng nói. Năm 2020, Vougioukas \cite{vougioukas2020} đã cải tiến phương pháp tạo sinh mặt và cập nhật thêm hành động chớp mắt, Lele Chen \cite{chen2020} cũng đưa ra phương pháp mới để tạo sinh mặt hiệu quả hơn, tự nhiên hơn với việc di chuyển của vùng đầu trên khung hình.
 
Nhìn chung, các nghiên cứu này đã đưa ra các kiến trúc mạng hiệu quả để tạo sinh khuôn mặt cũng như các phương pháp, lập luận và chứng minh tính hiệu quả của các kiến trúc mạng được đề xuất. Mặc dù các thông số của thử nghiệm đưa ra là khá tốt, các nghiên cứu của Vougioukas vẫn chưa thể tạo ra chuyển động của đầu, kết quả tạo sinh của Vougioukas đôi khi không giữ được đặc trưng của ảnh. Nghiên cứu của Lele Chen năm 2020 \cite{chen2020} đã tạo ra chuyển động cho phần đầu dựa trên tiếng nói, nhưng khuôn mặt được tạo sinh vẫn còn có thể bị nhận ra qua các phép thử Turing, và chuyển động của đầu đôi khi vẫn chưa được tự nhiên, mạng cũng có cấu trúc rất phức tạp và đòi hỏi nhiều tài nguyên tính toán để có thể huấn luyện.
