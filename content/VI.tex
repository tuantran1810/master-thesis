\chapter{Kết luận}
%Mô tả, bình luận ngắn gọn và đưa ra kết luận về kết quả nghiên cứu của luận văn và cách thức áp dụng thực tiễn. Đề ra các hướng nghiên cứu mở rộng cho Luận văn.

Với các cải tiến trong việc chuẩn hóa dữ liệu được đưa ra cho phương pháp tạo sinh ảnh từ bài báo \cite{chen2019}, nghiên cứu này đã đưa ra giải pháp để làm cho hình ảnh được tạo sinh trở nên chính xác hơn và phần nào cải thiện được độ sắc nét của hình ảnh trong tập dữ liệu GRID. Như đã được trình bày ở phần \ref{sec:comparision}, kết quả của nghiên cứu này là một sự cải thiện đáng kể so với mô hình gốc của tác giả Lele Chen. Tuy nhiên, độ sắc nét của hình ảnh vẫn còn thua kém khá xa so với các nghiên cứu cùng thời. So sánh với các nghiên cứu mới nhất ở thời điểm hiện tại, phương pháp được đề xuất vẫn còn khá đơn giản và có kết quả tạo sinh kém hơn hẳn và không miêu tả được chuyển động của đầu, cũng như không tái hiện được khung cảnh xung quanh. Tuy vậy, các nghiên cứu được nói đến ở trên dùng rất nhiều tài nguyên tính toán và khó có thể thực hiện trong điều kiện hiện tại của tác giả.

Trong những năm gần đây, việc tạo sinh hình ảnh gương mặt bắt đầu có nhiều ứng dụng hơn trong thực tiễn cuộc sống. Việc tạo ra phóng viên, biên tập viên truyền hình ảo đang trở nên cần thiết hơn bao giờ hết khi nó có thể tiết kiệm cực kì nhiều chi phí cho đài truyền hình. Chương trình thời sự có thể được truyền đi mà không cần hình ảnh nếu thiết bị đầu cuối có khả năng tạo sinh hình ảnh. Nhờ đó, thời gian truyền thông tin đi cũng nhanh hơn và ít tốn băng thông đường truyền hơn. Trong một số trường hợp đơn giản, tạo sinh hình ảnh gương mặt cũng giúp cho ngành điện ảnh tiết kiệm chi phí trong khâu hóa trang, kĩ xảo và thuê diễn viên. Ngoài ra, tạo sinh hình ảnh gương mặt cũng giúp cho những trò chơi điện tử trở nên chân thật hơn. Ngoài ra còn rất nhiều những ứng dụng thực tiễn khác trong cuộc sống từ các cuộc họp truyền hình, cho đến trợ lý ảo và nhân viên ảo cho cửa hàng.

Để mở rộng nghiên cứu này, ta cần ưu tiên nghiên cứu cách tạo sinh hình ảnh có độ nét cao hơn và nâng cao hơn nữa sự tương đồng giữa tiếng nói và hình ảnh, đồng thời là khẩu hình miệng người nói. Hiện tại, có một số phương pháp mới đang được nghiên cứu như việc mô hình hóa dữ liệu đầu vào trên không gian 3 chiều thay vì không gian 2 chiều. Cột mốc gương mặt có thể được trích xuất và dự đoán trong không gian 3 chiều. Dựa vào đó ta có thể đo các góc xoay của mặt và chuyển động của đầu, xác định chính xác các thành phần trên mặt như mắt, mũi, miệng và mô hình hóa phần đầu của người nói một cách cụ thể để tạo sinh hình ảnh rõ ràng, chính xác và hoàn chỉnh hơn. Phương pháp học thoáng qua (few-shot learning) cũng được áp dụng nhằm tạo sinh chuyển động gương mặt theo cách đặc trưng của người nói và cho kết quả tạo sinh có độ chân thật cao. Một số nghiên cứu mới gần đây cũng đã áp dụng phương pháp nêu trên như nghiên cứu mới của Lele Chen \cite{chen2020}, Ran Yi \cite{ranyi}
