
\section{\texorpdfstring{Nội dung dự kiến của luận văn}{Conclusion}}

Luận văn tốt nghiệp sẽ được chia thành các phần như sau:

\textit{Lời cam đoan của tác giả}: Cam đoan các công việc, thử nghiệm và kết quả được đưa ra trong luận văn là trung thực, khách quan.

\textit{Tóm tắt luận văn}: Trình bày ngắn gọn về cấu trúc của Luận văn, giới thiệu những điểm nhấn của Luận văn, kết quả, và các từ khóa đi kèm.

\textit{Mở đầu}: Nêu lý do chọn đề tài, mục đích, đối tượng và phạm vi nghiên cứu, ý nghĩa khoa học và ý nghĩa thực tiễn của đề tài.

\textit{Tổng quan tình hình nghiên cứu, mục tiêu và nhiệm vụ nghiên cứu}: Sơ lược, phân tích, đánh giá các công trình nghiên cứu nổi tiếng có liên quan đến đề tài. Nêu những vấn đề bức thiết cần phải giải quyết, chỉ ra những thiếu sót mà những nghiên cứu trước đây chưa giải quyết được.

\textit{Cơ sở lý thuyết}: Trình bày cơ sở lý thuyết, các lập luận, căn cứ khoa học được sử dụng trong Luận văn.

\textit{Phương pháp nghiên cứu}: Trình bày chi tiết về ý tưởng, các mô hình toán, các chứng minh nếu có. Đồng thời trình bày các bước thực hiện và khảo sát, kiểm nghiệm kết quả nghiên cứu. Mô tả kết quả nghiên cứu khi thử nghiệm với nhiều tập dữ liệu và những độ khó khác nhau.

\textit{Kết quả nghiên cứu}: Mô tả ngắn gọn các kết quả nghiên cứu, thực nghiệm. Bàn luận về điểm mạnh, điểm yếu của mô hình được xây dựng trong luận văn. So sánh kết quả thu được trong quá trình nghiên cứu, thực nghiệm của đề tài và đối chiếu với kết quả nghiên cứu, thực nghiệm của các tác giả khác một cách khách quan. Nêu lên điểm nổi bật, khác biệt của luận văn đối với các nghiên cứu khác.

\textit{Kết luận và hướng nghiên cứu mở rộng đề tài}: Mô tả, bình luận ngắn gọn và đưa ra kết luận về kết quả nghiên cứu của luận văn và cách thức áp dụng thực tiễn. Đề ra các hướng nghiên cứu mở rộng cho Luận văn.

\textit{Danh mục tài liệu tham khảo}: Trích dẫn các tài liệu được sử dụng trong Luận văn.
